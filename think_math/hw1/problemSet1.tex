\documentclass[12pt]{article}

\usepackage[utf8]{inputenc}
\usepackage[T2A]{fontenc}
\usepackage[english,russian]{babel}
\usepackage{amssymb}
\usepackage{graphicx}
\graphicspath{ {images/} }

\textwidth=431pt
\textheight=600pt
\hoffset=-30pt
\voffset=-30pt

\usepackage{graphicx}
\usepackage{amsmath}
\makeatletter
\renewcommand{\@oddhead}{%
	\vbox{%
		\hbox to \textwidth{\strut \textit{Think Math, Problem set 1, Usvyatsov Mikhail} \hfill }
		%\hbox to\textwidth{Лист\hfill Страница~\arabic{page}~из 2}
		\hrule
		\vspace{12pt}
	}}
	\renewcommand{\@oddfoot}{}
	\makeatother
	
	
\begin{document}
	\begin{center}
		\textbf{Problem set 1 \\
			DUE: Tue. September 8, 2015 \\}
	\end{center}
		
	\bigskip
	
	\textbf{Problem 1}		
		
	We know that all the polynoms complex roots have pairs, thus P(x) has a pair for $x_1= 1 - i\sqrt{7}$, that is $X_2 = 1 +  i\sqrt{7}$\\
	Also due to fundamental theorem of algebra P(x) = $(x - x_1)(x - x_2)(x - x_3)$. As far as we know $x_1$ and $x_2$ we can multiply them and get that P(x) = $(x^2 - 2x + 8)(x - x_3)$.\\
	Hence $x^3 - 2x^2 + 8x - x^2x_3 + 2xx_3 - 8x_3 =  -24+14x - 5x^2 +x^3$.\\
	$-8x_3 =  - 24$\\
	$x_3 = 3$
	
	\textbf{Problem 2}
	
	$x^3 - 2x^2 + 2x -1 = 0$\\
	In a nightmare I have found that $x_1$ = 1 is a solution of equation above. We can easily prove it: 1 - 2 + 2 - 1 = 0\\
	Then we can divide initial polynom by (x - 1) and according to fundamental theorem of algebra  we will have $(x - x_3)(x - x_2)$\\
	$(x - x_3)(x - x_2) = x^2 - x + 1$\\
	Thus $x_{2,3} = \dfrac{1 \pm i \sqrt{3}}{2}$
	
	\textbf{Problem 3}
	
	$x^3 - x^2 + \dfrac{x}{3} - \dfrac{1}{27} = 0$\\
	Lama in the hat had come to me and told that $x_1 = \dfrac{1}{3}$ is a root of previous equation.
	Then we can divide initial polynom by (x - $\dfrac{1}{3}$) and according to fundamental theorem of algebra  we will have $(x - x_3)(x - x_2)$\\
	$(x - x_3)(x - x_2) = x^2 - \dfrac{2}{3}x + \dfrac{1}{9} = (3x - 1)^2$\\
	Thus $x_{2,3} = x_1= \dfrac{1}{3}$
	
	\textbf{Problem 4}
	
	We know that geometry progression sum is $\dfrac{b_1(1 - q^n)}{1 - q}$.\\
	So, we can rewrite initial equation in another form:\\
	$\dfrac{1 - z^{2015}}{1 - z} = 0$\\
	We can check that z = 0 is not a root if the equation and thus we can multiply this equation by (1 - z) and so $1 - z^{2015} = 0$.\\
	$cos(2015\phi) + i sin(2015\phi) = 1$\\
	Defining $\phi = \dfrac{2\pi k}{2015}$, where $k \in N$\\
    z = $cos \phi + i sin \phi$ is a root of initial equation
    
    \textbf{Problem 5}
    
    P(x) = $x^3 - 2^{-\dfrac{2}{3}}zx + 1$.\\
    This polynom has 3 roots. It means that we can rewrite it in the following form.\\
    P(x) = $(x - x_1)(x - x_2)^2$\\
    We can do it because $x_2 = x_3$.\\
    Thus, opening brackets in a new form of our polynom we get:\\ 
    P(x) = $x^3 - x^2(2x_2 + x_1) + x(x_2^2 + 2x_1x_2) - x_1x_2^2$\\
    From initial form and a new one we can get a system:\\
    \begin{equation*}
	    \begin{cases}
		    x_1x_2^2 = 1\\
		    2x_2+x_1 = 0\\
		    x_2^2+2x_1x_2 = -2^{-\dfrac{2}{3}}z
	    \end{cases}
    \end{equation*}
    From this system we can get that:\\
    \begin{equation*}
	    \begin{cases}
		    x_1 = - 2x_2\\
		    x_2^3 =-\dfrac{1}{2}\\
		    z = \dfrac{x_2^2+2x_1x_2}{-2^{-\dfrac{2}{3}}}
	    \end{cases}
    \end{equation*}
    Hence:
    \begin{equation*}
	    \begin{cases}
		    x_1 = - 2x_2\\
		    x_2^3 =-\dfrac{1}{2}\\
		    z = 2^{-\dfrac{2}{3}} \cdot 3x_2^2 
	    \end{cases}
    \end{equation*}
    From this system we can get that: z should be equal to -3.
    
    \textbf{Problem 6}
    
    cosh(x) = $\dfrac{e^x + e^{-x}}{2}$\\
    $cosh(log(x + \sqrt{x^2 - 1})) = \dfrac{e^{log(x + \sqrt{x^2 - 1})} + e^{-log(x + \sqrt{x^2 - 1})}}{2}$\\
    Assuming that logarithm base is уб we can derive that initial function is equal to:
    $\dfrac{x + \sqrt{x^2 - 1} + \dfrac{1}{x + \sqrt{x^2 - 1}}}{2}=x$
    
    \textbf{Problem 7}
    
    $sinh(x) = \dfrac{e^x - e^{-x}}{2}$\\
    $cosh^2(t) - sinh^2(t) = 1$\\
    By definition of hyperbolic functions:\\
    $cosh^2(t) - sinh^2(t) = \dfrac{(e^t + e^{-t})^2 - (e^t - e^{-t})^2}{4} = \dfrac{e^{2t} + 2 + e^{-2t} - e^{2t}+2-e^{-2t}}{4} = 1$\\
    Nothing will change for any t because it doesn't depend on it.
    
    \textbf{Problem 8}
    
    Using L'Hopital's rule several times we can find that:\\
    $\lim_{x \to 0} \dfrac{i sin(-ix) + sin(x) - 2x}{log(1 + x^5)} = \lim_{x \to 0} \dfrac{(cos(ix) + cos(x) - 2)(1 + x^5)}{5x^4} =$\\
    $\lim_{x \to 0} \dfrac{(cos(ix) + cos(x) - 2)\cdot 5x^4 + (-isin(ix) - sin(x))(1 + x^5)}{20x^3}=$\\
    $\lim_{x \to 0} \dfrac{(cos(ix) + cos(x) - 2)\cdot 20x^3 + (-isin(ix) - sin(x))\cdot 5x^4 +}{}\\ \dfrac{+ (-isin(ix) - sin(x))\cdot 5x^4 + (cos(ix) - cos(x))(1 + x^5)}{60x^2}=$\\   
    $\lim_{x \to 0} \dfrac{(cos(ix) + cos(x) - 2)\cdot 60x^2 + (-isin(ix) - sin(x))\cdot 20x^3 + (-isin(ix) - sin(x))\cdot 40x^3 + }{}\\
    \dfrac{+ (cos(ix) - cos(x))\cdot 10x^4 + (cos(ix) - cos(x))\cdot 5x^4 + (-isin(ix) + sin(x))(1 + x^5)}{120x}$ = \\
    $\lim_{x \to 0} \dfrac{(cos(ix) + cos(x) - 2)\cdot 120x + (-isin(ix) - sin(x))\cdot 60x^2 + (-isin(ix) - sin(x))\cdot 180x^3}{}\\ \dfrac{ + (cos(ix) - cos(x))\cdot 60x^3 + (cos(ix) - cos(x))\cdot 60x^3 + (-isin(ix) + sin(x))\cdot 15x^4 +}{}\\ \dfrac{ + (-isin(ix) + sin(x))\cdot 5x^4 + (cos(ix) + cos(x))(1 + x^5)}{120} = \dfrac{2}{120} = \dfrac{1}{60}$\\ 
    
    \textbf{Problem 9}
    
    By definition of cosh we can derive that:\\
    cosh(in arccos(x)) = $\dfrac{e^{in \cdot arccos(x)} + e^{-in \cdot arccos(x)} }{2}$\\
    According to Euler's formula:\\
	$\dfrac{e^{in \cdot arccos(x)} + e^{-in \cdot arccos(x)} }{2} = \dfrac{cos(n\cdot arccos(x)) + i sin(n\cdot arccos(x)) +}{}\\ \dfrac{+ cos(n\cdot arccos(x)) -  i sin(n\cdot arccos(x))}{2} = cos(n\cdot arccos(x))$\\
	When n = 0:\\
	$cos(n\cdot arccos(x) = cos(0) = 1$\\
	When n = 1:\\
	$cos(n\cdot arccos(x) = cos(arccos(x) = x$\\
	When n = 2:\\
	$cos(n\cdot arccos(x) = cos(2\cdot arccos(x) = 2cos^2(arccos(x)) - 1 = 2x^2 - 1$\\
	When n = 3:\\
	$cos(n\cdot arccos(x) = cos(3\cdot arccos(x) = 4cos^3(arccos(x)) - 3 cos(arccos(x)) = 4x^3 - 3x$\\
 
    \textbf{Problem 10}\\
    z = a + ib\\
    w = с + id\\
    |z-w| = $\sqrt{(a-c)^2 + (b-d)^2}$\\
    |z| = $\sqrt{a^2 + b^2}$\\
    |w| = $\sqrt{c^2 + d^2}$\\
    ||z| - |w|| = $|\sqrt{a^2 + b^2} - \sqrt{c^2 + d^2}|$\\
    Now we can rewrite :
    $ |z-w| \geq ||z| - |w|| $\\
    as:\\
    $\sqrt{(a-c)^2 + (b-d)^2} \geq | \sqrt{a^2 + b^2} - \sqrt{c^2 + d^2	}|$\\
    $a^2 + b^2 +c^2 +d^2 - 2ac - 2bd \geq a^2 + b^2 + c^2 +d^2 - 2\sqrt{(a^2 + b^2)(c^2 + d^2)}$\\
    $ \sqrt{(a^2 + b^2)(c^2 + d^2)} \geq  ac + bd $\\
    $ (a^2 + b^2)(c^2 + d^2) \geq  (ac)^2 + (bd)^2 + 2(ac)(bd) $\\
    $ (ac)^2 + (ad)^2 + (bc)^2 + (bd)^2 \geq  (ac)^2 + (bd)^2 + 2(ac)(bd) $\\
    $ (ad)^2 + (bc)^2 \geq  2(ac)(bd) $\\
    $ \dfrac{(ad)^2 + (bc)^2}{2} \geq  \sqrt{(ac)^2(bd)^2} $\\
    The last equality always holds, QED.
    
    \textbf{Problem 11}\\
    $\dfrac{\partial f}{\partial x_1} = \dfrac{\partial f_1}{\partial x_1} + i \dfrac{\partial f_2}{\partial x_1}$\\
    $\dfrac{\partial f}{\partial x_2} = \dfrac{\partial f_1}{\partial x_2} + i \dfrac{\partial f_2}{\partial x_2}$\\
    $\dfrac{\partial^2 f_1}{\partial x_1^2} + \dfrac{\partial^2 f_1}{\partial x_2^2}  = \dfrac{\partial}{\partial x_1} \left( \dfrac{\partial f_2}{\partial x_2} \right) - \dfrac{\partial}{\partial x_2} \left( \dfrac{\partial f_2}{\partial x_1} \right) = 0 $
    
    \textbf{Problem 12}
    
    $x\cdot e^{2\pi i} = x (cos(2\pi) + i sin(2 \pi)) = x$, QED.\\
    Due to the fact, that $x\cdot e^{2\pi i} = x$, $log(x e^{2\pi i }) - log(x) = log(x) - log(x) = 0$
\end{document}